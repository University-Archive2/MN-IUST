%% Last Update: 24 Jul 2019
%

%% =================================================================
%% توسط دستور \myData می توانید تاریخ و ساعت را وارد متن خود کنید. 
\newcommand{\myData}{
	\newcount \hour
	\newcount \min
	\newcommand*{\timeoftoday}{%
		\hour\time\divide\hour 60 ساعت \the\hour{}
		\min\time\multiply\hour 60 \advance\min-\hour
		\ifnum\min=0
		\else
		و
		\the\min{} دقیقه
		\fi}
	\today{} در \timeoftoday
} %M


%% =================================================================
%برای تعریف دوباره یک operator در LateX
\makeatletter
\newcommand\RedeclareMathOperator{%
  \@ifstar{\def\rmo@s{m}\rmo@redeclare}{\def\rmo@s{o}\rmo@redeclare}%
}
% this is taken from \renew@command
\newcommand\rmo@redeclare[2]{%
  \begingroup \escapechar\m@ne\xdef\@gtempa{{\string#1}}\endgroup
  \expandafter\@ifundefined\@gtempa
     {\@latex@error{\noexpand#1undefined}\@ehc}%
     \relax
  \expandafter\rmo@declmathop\rmo@s{#1}{#2}}
% This is just \@declmathop without \@ifdefinable
\newcommand\rmo@declmathop[3]{%
  \DeclareRobustCommand{#2}{\qopname\newmcodes@#1{#3}}%
}
\@onlypreamble\RedeclareMathOperator
\makeatother


%% =================================================================
%%% Math
%اضافه کردن برخی دستورات ریاضیاتی که وجود ندارد. 
\RedeclareMathOperator*{\min}{min}

% اضافه کردن نماد استقلال به علایم ریاضی
\newcommand{\Perp}{\perp \! \! \! \perp}
\newcommand{\independent}{\protect\mathpalette{\protect\independenT}{\perp}} \def\independenT#1#2{\mathrel{\rlap{$#1#2$}\mkern2mu{#1#2}}}


%% =================================================================
% این دستور به این خاطر تعریف می‌شود، که اگر بخواهیم مثلا یک کلمه در واژه نامه را Index‌کنیم، مثلا بنویسیم، ‎\index{\glspl{Water}‎}‎، این دستور index درست عمل نمی‌کند. نویسنده بسته glossaries در این باره این طور گفته است.
%If you inspect the .idx file you will see that it contains the following:
%\indexentry{\glsentryplural{StrictlyStable}|hyperpage}{1}
%\index doesn't expand its argument when writing to the .idx file and since xindy doesn't understand (La)TeX commands the index won't be correctly sorted. This is a feature of \index and is not connected with glossaries.
%You could try something like
%\expandafter\index\expandafter{\glsentryplural{StrictlyStable}}
% توسط دستور تعیین شده می‌توان این مشکل را حل کرد. دقت کنید که این دستور می‌بایست حتما برای واژه‌نامه ها فقط تعریف شود.

%\let\oldindex\index
%\renewcommand{\index}[1]{
%\ifthenelse{\equal{\glsentrytype{#1}}{english}}
	%{\expandafter\oldindex\expandafter{\glsentrytext{#1}}}
	%{ \ifthenelse{\equal{\glsentrytype{#1}}{acronym}}
		%{\expandafter\oldindex\expandafter{\glsentrytext{#1}}}
		%{\oldindex{#1}}
	%}
%}


%% =================================================================
% حل شدن مشکل part
% تعریف یکسری دستور به منظور حل مشکل قرار دادن part در متن و آمدن آن در فهرست مطالب. 
\makeatletter
\ifthenelse{\equal{\@docStyle}{Report}}
{%
\renewcommand*\l@part[2]{
	\ifnum \c@tocdepth >-2\relax
	\addpenalty{-\@highpenalty}
	\addvspace{2.25em \@plus\p@}
	\setlength\@tempdima{3em}
	\begingroup
	\parindent \z@ \if@RTL\leftskip\else\rightskip\fi \@pnumwidth
	\parfillskip -\@pnumwidth
	{\leavevmode
		\large \bfseries\textcolor{blue}{بخش} #1
		\hfil \hb@xt@\@pnumwidth{\hss #2}}\par
	\nobreak
	\global\@nobreaktrue
	\everypar{\global\@nobreakfalse\everypar{}}
	\endgroup
	\fi}

\def\@chapter[#1]#2{\ifnum \c@secnumdepth >\m@ne
	\if@mainmatter
	\refstepcounter{chapter}%
	\typeout{\@chapapp\space\thechapter.}%
	\addcontentsline{toc}{chapter}%
	{\@chapapp~\protect\numberline{\arabic{chapter}}~~~ #1}%
	\else
	\addcontentsline{toc}{chapter}{#1}%
	\fi
	\else
	\addcontentsline{toc}{chapter}{#1}%
	\fi
	\chaptermark{#1}%
	%\addtocontents{lof}{\protect\addvspace{10\p@}}%
	%\addtocontents{lot}{\protect\addvspace{10\p@}}%
	\if@twocolumn
	\@topnewpage[\@makechapterhead{#2}]%
	\else
	\@makechapterhead{#2}%
	\@afterheading
	\fi}
%\renewcommand*\l@section{\@dottedtocline{1}{4.5em}{2.3em}}
%\renewcommand*\l@subsection{\@dottedtocline{2}{6.5em}{3.2em}}
}{}
\makeatother


%% =================================================================
% این دستور به این خاطر تعریف می‌شود، که اگر بخواهیم مثلا یک کلمه در واژه نامه را Index‌کنیم، مثلا بنویسیم، ‎\index{\glspl{Water}‎}‎، این دستور index درست عمل نمی‌کند. نویسنده بسته glossaries در این باره این طور گفته است.
%%%If you inspect the .idx file you will see that it contains the following:
%%%\indexentry{\glsentryplural{StrictlyStable}|hyperpage}{1}
%%%\index doesn't expand its argument when writing to the .idx file and since xindy doesn't understand (La)TeX commands the index won't be correctly sorted. This is a feature of \index and is not connected with glossaries.
%%%You could try something like
%%%\expandafter\index\expandafter{\glsentryplural{StrictlyStable}}
% توسط دستور تعیین شده می‌توان این مشکل را حل کرد. دقت کنید که این دستور می‌بایست حتما برای واژه‌نامه ها فقط تعریف شود. 
\newcommand{\indexglspl}[1]{\expandafter\index\expandafter{\glsentryplural{#1}}}
\newcommand{\indexgls}[1]{\expandafter\index\expandafter{\glsentryname{#1}}}
