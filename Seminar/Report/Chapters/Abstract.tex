\chapter{چکیده}
یکی از اجزای اصلی در شبکه‌های تلفن همراه، ناحیه دسترسی رادیویی است و سازمان 
\lr{O-RAN Alliance}
با شروع و استانداردسازی معماری جدیدی تحت عنوان
\lr{O-RAN}
راه جدیدی را آغاز کرده که مدیریت و بهینه‌سازی شبکه‌های تلفن همراه را متحول کرده‌است. 

در این معماری با جدا کردن قسمت‌های مختلف ناحیه دسترسی رادیویی، استفاده از مجازی‌سازی و اجزای داده‌محور مختلف امکان مدیریت هوشمند و خودمختار به ناحیه رادیویی داده شده‌است. 

در این گزارش، بخش‌های مختلف این معماری به تفکیک بررسی شده‌اند و اجزای مختلفی که به هوشمندی و داده‌محوری این قسمت کمک کرده‌اند مورد بحث قرار گرفته‌اند.

واژه‌های کلیدی: ناحیه‌ی دسترسی رادیویی، شبکه‌های تلفن‌همراه، یادگیری ماشین، 
\lr{O-RAN}