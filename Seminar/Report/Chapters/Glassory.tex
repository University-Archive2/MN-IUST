\chapter{شرح مختصری بر واژه نامه}
در
\LaTeX
نیاز ندارید که هیچ کلمه‌ی ای را به زبان انگلیسی بنویسید. شما تنها نیاز است که ترجمه کلمات را در فایل واژه نامه خود قرار دهید و در متن خود، آن‌ها را فراخوانی نمایید. برای مثال، فرق بین لغت 
\gls{Wireless Communication}
و 
\glspl{Wireless Communication}
که در فایل 
\lr{MyWords}
تعریف شده است را ببینید. با دستور 
\lr{ (back slash) gls{Wireless Communication}}
شما نسخه‌ی مفرد ترجمه‌ی این لغت را فراخوانی می‌کنید و با دستور 
\lr{(back slash) glspl{Wireless Communication}}،
نسخه‌ی جمع را فراخوانی می‌کنید. همچنین برای فهرست 
\glspl{Abbrevation}
نیز همانند چند 
\gls{Abbrevation}
تعریف شده در 
\lr{My Abbrevation}
عمل نمایید. 
به عنوان مثل، 
\gls{MANET}
یا 
\gls{LTE}.
\begin{note}
	تمام لغت‌های خود را در فایل 
	\lr{MyWords}
	تعریف کرده و همچنین تمام 
	\glspl{Abbrevation}
	خود را در فایل 
	\lr{MyWords}
	تعریف نمایید. برای راحتی جست و جوی لغات، تمام لغات را در فایل 
\lr{All words}
	برای شما چاپ کرده ام. 
\end{note}
\begin{warning}
	دقت کنید که چاپ واژه نامه با کامپایل فایل انجام نمی‌شود. مراحلی دارد که برای یادگیری آن، نوشتار دکتر دیانت  در سایت پارسی لاتک و با جست و جوی عبارت راهنمای ایجاد واژه‌نامه در لاتک را مطالعه نمایید.
\end{warning}
موفق باشید. 
امیرحسین جلیلوند.