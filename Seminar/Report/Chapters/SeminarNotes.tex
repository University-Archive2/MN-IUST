\chapter{ نکاتی در مورد سمینار درس}
با سلام و آرزوی  سلامتی برای شما دانشجویان عزیز. این نوشتار کوتاه برای راهنمایی شما در مورد سمینار و گزارش سمینار در درس 
\glspl{Wireless Communication}
نوشته شده است. هدف این سمینار آشنایی شما دانشجویان محترم با یکی از موضوعات روز دنیای
\glspl{Wireless Network}
از طریق بررسی  و واکاوی یک مقاله‌ی مروری معتبر و تحقیق در مورد آن موضوع و یافتن جدید ترین دست آورد‌های محققین در آن حوزه مشخص می‌باشد. بدین‌سان در این نوشتار تلاش خواهیم کرد تا آن‌چه در بخش سمینار نیاز است که بدانید را خدمت شما دانشجویان محترم توضیح دهیم.

\begin{goal}{نکته}
	سعی می‌شود در این نوشتار نکاتی که باعث کیفیت مناسب در کار شما شود را ذکر کنیم. همچنین نکاتی را با ذکر  عبارت (تاثیر مثبت) بیان خواهیم نمود. به این معنا که انجام این نکات در انجام کار الزامی نبوده، اما تاثیر مناسبی در کیفیت کار و نتایج بررسی‌ها خواهد داشت.
\end{goal} 

بدون هیچ مقدمه‌ای به بیان آن چه در این درس از شما دانشجویان محترم انتظار می‌رود،  می‌پردازیم. در این درس یک مقاله‌ی مروری برای شما در نظر گرفته‌ شده است. مقالات در موضوعات وسیعی از حوزه‌ی 
\gls{Wireless Communication}
گردآوری شده اند. در
(\autoref{L1}
،
\autoref{L2}
و
\autoref{L3})
فرایند انجام کار را خدمت شما ارائه می‌کنیم. 

\section{انتخاب موضوع}\label{L1}
فرایند انتخاب موضوع به مدت 10 روز از تاریخ 12 اسفند 1400 تا تاریخ 24 اسفند 1400 به طول خواهد انجامید. بدین صورت که دانشجویان محترم، می‌توانند لیست موضوعات و افراد تخصیص یافته  را بررسی می‌نمایند. این لیست در مخزن اسلاید‌های درس موجود است. بعد از بررسی عنوان‌های انتخاب نشده  توسط دانشجویان،   هر دانشجو، تعداد سه موضوع مورد علاقه خود را به ترتیب اولویت از طریق ایمیل رسمی درس با عنوان ایمیل (انتخاب موضوع سمینار) ارسال می‌نماید. ایمیل‌های دریافتی در بازه‌ی مجاز، توسط کمک مدرسین بررسی می‌شود و اگر موضوعات دانشجویان توسط دانشجویان دیگر انتخاب نشده‌ باشد، ایمیل تایید ثبت موضوع برای ایشان ارسال می‌شود. دریافت این ایمیل به منزله تایید موضوع و ثبت نهایی در فرم موضوعات سمینار است.  پس از زمان تخصیص داده‌شده،  چنانچه دانشجویان محترم انتخاب موضوع خود را فراموش نمایند، موضوعات توسط گروه کمک مدرسین و به صورت تصادفی برای ایشان ثبت می‌شود.  
\section {ارائه‌ی شفاهی}\label{L2}
این بخش حدود 30 درصد از نمره‌ی سمینار را به خود اختصاص می‌دهد. در ارائه‌ی شفاهی، شما نیاز است که اسلاید‌های طراحی شده‌ی خود را ارائه دهید. همچنین به سوالات حاضرین در جلسه به صورت شفاهی پاسخ دهید (در صورت برقراری جلسه‌ی آنلاین ارائه). در مورد جلسه‌ی ارائه از طریق گروه و سامانه‌ی 
\href {http://lms.iust.ac.ir/}{سامیا}
به شما اطلاع رسانی خواهد شد. 
\begin{ntpoint}
	در مورد طراحی اسلاید‌ها شما مختار هستید که از روش‌های متنوع مانند:
	\begin{itemize}
		\sci
		نرم افزار پاورپوینت،
		\sci 
		\LaTeX (\lr{Beamer})،
		\sci 
		نرم افزار
		\lr{Prezi}
	\end{itemize}
	و غیره استفاده کنید. 
\end{ntpoint}
اگر از نرم افزار  پاور پوینت استفاده می‌کنید، نیاز است که اسلاید‌های خود را صداگذاری کنید. همچنین برای اسلاید‌های طراحی شده توسط بسته‌ی 
\lr{Beamer}
و 
\lr{Prezi}
نیز صداگذاری و تدوین لازم است. 
وجود فهرست مطالب، مقدمه، نتیجه گیری، فهرست ارجاعات و اسلاید پرسش و پاسخ در اسلاید‌ها الزامی است. شما در تولید محتوای ارائه خود با هر روشی که راحت هستید، مختارید. هرچقدر کار شما زیبا تر و حرفه‌ای تر باشد، نتیجه نیز بهتر خواهد بود.
\subsection{زمان ارائه}
محدودیت زمان (حداکثر 25 دقیقه) را مد نظر داشته باشید. مهلت ارسال فایل‌های صداگذاری شده و محتوای‌ تولید شده‌ی شما از طریق سامانه‌ی 
	\href{http://lms.iust.ac.ir/}{سامیا}
به اطلاع شما خواهد رسید. احتمالا هم در مورد حداکثر حجم فایل ارسالی نکاتی به شما گفته خواهد شد.
\section{گزارش‌  کتبی سمینار}\label{L3}
بخش گزارش کتبی نیز حدود 70 درصد از نمره‌ی سمینار را به خود اختصاص می‌دهد. نکته‌ی اول این که استفاده از سیستم حروف چینی 
\LaTeX
و بسته‌ی 
\XePersian
در گزارش‌های شما الزامی است و استفاده از نرم افزار
\lr{Word}
امتیازی ندارد. در گارگاه 
\TeX
که توسط  استاد محترم و کمک مدرسین این درس برگزار شد، به خوبی استفاده از سیستم حروف چینی 
\LaTeX
به شما دانشجویان محترم آموزش داده شد. همچنین  استفاده از قالب  فارسی 
\href{https://github.com/abodin/Boostan}{بوستان}
و نکات مربوط به تولید واژه نامه و فهرست اختصارات به صورت اتوماتیک به خوبی عنوان شد.
در این قسمت برخی از نکات مهم را به دانشجویان محترم متذکر می‌شویم. همان‌طور گه گفته شد، نوشتن متن و تحویل گزارش با سیستم حروف‌چینی 
\LaTeX
و بسته‌ی 
\XePersian
الزامی است. با این حال، استفاده از قالب
\href{https://github.com/abodin/Boostan}{بوستان}
الزامی نیست چرا که ممکن است قالب دیگری از سایت
\href{http://www.parsilatex.com/wiki/}{پارسی لاتک}
توسط دانشجویان محترم استفاده شود. 
اما در قالب‌های 
\XePersian
استفاده شده توسط دانشجویان،
رعایت ویژگی‌هایی نظیر:
\begin{itemize}
	\tick
	واژه‌نامه‌ی خودکار،
	\tick
	فهرست اختصارات خودکار،
	\tick 
	فهرست جداول و تصاویر خودکار
\end{itemize}
الزامی است. 

دقت کنید که از آن‌جایی فایل‌های گزارش شما توسط مصححین بررسی می‌شود، ارسال کلیه‌ی فایل‌ها اعم از فایل واژه نامه و فهرست اختصارات، تصاویر و فایل مربوط به مراجع الزامی است. در مورد تصاویر، استفاده از تصاویر مقاله‌ی مروری و سایر مراجع (که به زبان انگلیسی هستند) با فرمت وکتور لازم و کافی است. اما بازترسیم تصاویر مقالات به زبان فارسی با فرمت وکتور و با استفاده از نرم افزار‌های مختلف نظیر:  
\begin{itemize}
\X\lr{Visio}،
\X
بسته‌ی 
\lr{Tikz}،
\X
\lr{CorelDraw}
\end{itemize}
و سایر نرم افزار‌ها، تاثیر مثبت دارد. 
\subsection{چالش واژه‌‌نامه}
برای حل چالش‌ تولید و استفاده از واژه‌نامه،  راهکار مدنظر ما به این صورت است که در این نیم‌سال، تمام دانشجویان برای داشتن واژه‌نامه یکسان، در یک پروژه در گیتلب عضو  می‌شوند. لازم به ذکر است که متعاقبا مراحل عضویت در پروژه گیتلب کلاس به اطلاع خواهد رسید. فایل واژه نامه هر گزارش، باید در گیتلب قرار بگیرد، بدین صورت که ابتدا، واژه‌های تولید‌شده‌ در سنوات گذشته، در اختیار دانشجویان قرار می‌گیرد و هر دانشجو علاوه بر استفاده از آن لغات، باید فایل لغات اضافی گزارش خود را نیز در پروژه گیتلب اضافه نماید. 

\subsection{چالش مدیریت منابع}
استفاده از نرم‌افزار‌های مدیریت منابع به منظور دسته‌بندی  مقالات و نکات مربوط به ارجا‌ع‌ها برای تحقیق و پژوهش شما در مقطع پیش‌رو بسیار مفید است. توصیه می‌شود که از یکی از نرم افزار‌های مناسب مدیریت منبع به مانند
	\href{https://www.mendeley.com/download-desktop-new/}{\lr{Mendeley}}
	استفاده کنید. تمام نکات مربوط به ساختار مراجع به سادگی با بهره‌گیری از این نرم افزار‌ها درست می‌شود.
	
\subsection{لزوم رعایت اصول نگارشی}
در مورد متن گزارش خود، حتما اصول نگارشی، ارجاعات و سایر اصول استاندارد را رعایت کنید. توصیه می‌کنیم که متن خود را (به صورت دو به دو) با دوست خود به اشتراک بگذارید و اشکالات همدیگر را رفع نمایید. این کار تاثیر بسیار خوبی روی قدرت نویسندگی شما خواهد گذاشت. دقت کنید که متن شما باید عاری از لغات تخصصی لاتین باشد (مگر در موارد اضطراری). بنابراین لازم است لغات تخصصی را در واژه‌نامه قرار دهید. این قدرت باعث می‌شود که هر لحظه‌ای که اراده نمایید، بتوانید ترجمه‌ی لغت مربوطه را در متن خود تغییر دهید! 
\subsection{  تعداد صفحات گزارش}
با توجه به این که هدف این پروژه بررسی یک موضوع تحقیقاتی وکار‌های پژوهشی اخیر در آن موضوع است، طبیعتا می‌توان صفحات زیادی را مشغول نوشتن بود. اما نکته‌ای که مد نظر ما است، حجم گزارش نیست! بیشتر یادگیری و انتقال موضوع برای ما اهمیت دارد. اما به صورت یک قاعده‌ی کلی در مورد این گزارش، حداقل متن (بدون در نظر گرفتن فهرست‌ مطالب، تصاویر، جداول و واژه‌نامه و مراجع) 30 صفحه تعیین شده.  نوشتن گزارش بیش از 30  صفحه به شرطی که صفحات اضافی موجبات غنای بیشتر موضوع باشد (نه صرفا افزایش حجم)، تاثیر مثبت دارد. با این حال تاکید بر نوشته‌ی کامل و استاندارد (نه الزاما طولانی) است. 