\Question{روند تخصیص فرکانس}
{
	هر قسمت‌‌ از طیف فرکانسی باید به کاربرد خاصی اختصاص یابد. تخصیص فرکانسی باید در لایه‌های مختلفی مانند کشوری، منطقه‌ای و جهانی انجام شود.
	
	\lr{ITU}\LTRfootnote{\lr{International Telecommunication Union}}
	به عنوان سیاست‌گذار اصلی و جهانی، حق حاکمیت هر کشوری برای تنظیم ارتباطات را به رسمیت می‌شناسد و آن را به کشورها واگذار کرده است. این نهاد، هر ۴ سال یک بار نشستی ترتیب می‌دهد تا سیاست‌های کلی را به صورت جهانی برای همگان تبیین کند.
	
	تخصیص فرکانس‌ها باید به صورت بهینه‌ای باشد و به تمامی کاربری‌ها از جمله تکنولوژی‌های جدید توجه کند. در ضمن این تخصیص باید به نحوی باشد که باعث تداخل فرکانسی نشود پس لازم است که در مرزها، بین کشورها و تنظیم‌کنندگان هماهنگی صورت گیرد.
}