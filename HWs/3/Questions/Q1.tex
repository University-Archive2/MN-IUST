\Question{
	به روزسانی منطقه‌ی مکانی در نسل ۴
}{
در نسل‌های قبلی با استفاده از 
\lr{LAC}\LTRfootnote{Location Area Code}
منطقه‌ی مکانی فرد ذخیره می‌شد و در زمان نیاز عملیات‌های 
\lr{Paging}
در سلول‌های موجود در آن
\lr{LAC}
انجام می‌شد. اما در نسل جدیدتر یعنی در نسل ۴، این مفهوم جای خود را به 
\lr{TAC}\LTRfootnote{Tracking Area} 
داده است. در حالت جدید، یک سلول می‌تواند در چندین 
\lr{TA} 
قرار گیرد یعنی برخلاف گذشته قابلیت همپوشانی وجود دارد. هر کاربر یک لیستی از این
\lr{TA}ها
دارد و تا وقتی که در سلول‌هایی که در این لیست موجود است حرکت می‌کند آپدیت مکانی‌ای برای او ثبت نمی‌شود و زمانی که به سلول‌های خارج این لیست برود، یک آپدیت مکانی برای اون ثبت خواهد شد. البته بعضی از این آپدیت‌ها هم به صورت پریودیک صورت می‌گیرد و به خاطر جلوگیری از ازدحام همه‌ی کاربران آن‌ها را همزمان دریافت نمی‌کنند.
}